\documentclass{article}
\usepackage{arxiv}

\usepackage[utf8]{inputenc}
\usepackage[english, russian]{babel}
\usepackage[T2A, T1]{fontenc}
\usepackage{url}
\usepackage{booktabs}
\usepackage{amsfonts}
\usepackage{nicefrac}
\usepackage{microtype}
\usepackage{lipsum}
\usepackage{graphicx}
\usepackage{natbib}
\usepackage{doi}



\title{Ускоренные методы нулевого порядка в гладкой выпуклой стохастической оптимизации}

\author{
	Хафизов Фанис \\
	\texttt{khafizov.fa@phystech.edu} \\
	%% examples of more authors
	\And
	Богданов Александр \\
	\texttt{bogdanov.ai@phystech.edu} \\
	\And
	Безносиков Александр \\
	\texttt{beznosikov.an@phystech.edu}
	%% \AND
	%% Coauthor \\
	%% Affiliation \\
	%% Address \\
	%% \texttt{email} \\
	%% \And
	%% Coauthor \\
	%% Affiliation \\
	%% Address \\
	%% \texttt{email} \\
	%% \And
	%% Coauthor \\
	%% Affiliation \\
	%% Address \\
	%% \texttt{email} \\
}
\date{}

\renewcommand{\shorttitle}{Ускоренные методы нулевого порядка}
\renewcommand{\undertitle}{}
%%% Add PDF metadata to help others organize their library
%%% Once the PDF is generated, you can check the metadata with
%%% $ pdfinfo template.pdf
\hypersetup{
pdftitle={Ускоренные методы нулевого порядка в гладкой выпуклой стохастической оптимизации},
pdfsubject={Стохастическая оптимизация},
pdfauthor={Хафизов Ф.А., Богданов А.И., Безносиков А.Н.},
pdfkeywords={Методы нулевого порядка},
}

\begin{document}
\maketitle

\begin{abstract}
Данная работа посвящена задачи оптимизации черного ящика. В такой постановке у нас нет доступа к градиенту целевой функции $f(x)$, из-за чего нам надо как-то его оценивать. Рассматривается безградиентный метод JAGUAR, использующий информацию о предыдущих вызовах и требует $\mathcal{O}(1)$ оракульных вызовов. Мы применяем эту аппроксимацию в ускоренном методе градиентного спуска и докажем его сходимость для выпуклой задачи оптимизации. Также сравним метод JAGUAR с другими известными способами на экспериментах.

\end{abstract}


\keywords{методы нулевого порядка, стохастическая оптимизация}

\section{Введение}

\bibliographystyle{unsrtnat}
%\bibliography{references}

\end{document}
